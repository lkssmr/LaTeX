\documentclass[11pt]{beamer}
\usepackage[utf8]{inputenc}
\usepackage[ngerman]{babel}
\usepackage[T1]{fontenc}
\usepackage{lmodern}
\usepackage[]{babel}
\usetheme{Copenhagen}
\definecolor{theme}{HTML}{383e79}
\definecolor{term}{HTML}{009b8e}
\definecolor{termref}{HTML}{008080}
\definecolor{correct}{HTML}{383e79}
\usepackage{tikz, pgfplots, amsmath, amsthm, amssymb}
\usetikzlibrary{patterns}

\begin{document}
	\author{Lukas Semrau \\ \href{mailto:mail@lukas-semrau.de}{\texttt{mail@lukas-semrau.de}}}
	\title{Einführung in \LaTeX und Tik\textit Z}
	%\subtitle{}
	%\logo{}
	\institute{Bernhard-Strigel-Gymnasium}
	%\date{}
	%\subject{}
	%\setbeamercovered{transparent}
	%\setbeamertemplate{navigation symbols}{}
	
	\begin{frame}
		Eine lineare Funktion besitzt den Funktionsterm $f:x\mapsto mx+t$. Die Steigung $m$ wird durch die Gleichung \begin{align}
			m&=\frac{\Delta y}{\Delta x} = \frac{f(b)-f(a)}{b-a}
		\end{align} definiert.
		
		\begin{center}
			\begin{tikzpicture}
				\begin{axis}[
					x=1cm,y=1cm,
					axis lines=middle,
					ymajorgrids=true,
					xmajorgrids=true,
					xmin=-1,
					xmax=5,
					ymin=-1,
					ymax=4,
					xtick={2,4},
					xticklabels={\color{theme} $a$, \color{theme} $b$},
					ytick={2,3},
					yticklabels={\color{theme} $f(a)$, \color{theme} $f(b)$}]
					%Below the red parabola is defined
					\addplot [
					domain=1.9:5, 
					samples=100, 
					color=black,
					]{1/2*x+1};
					\addplot [
					domain=-1:1.05, 
					samples=100, 
					color=black,
					]{1/2*x+1};
					
					\draw[color = theme,fill = theme!30, line width = 1pt] (axis cs:2,2) -- (axis cs:4,2) -- (axis cs:4,3) -- (axis cs:2,2);
					\draw [fill = black, color = black] (axis cs:4,3) circle (1pt) node[above left]{$B(b\mid f(b)$};
					
					\draw [fill = black, color = black] (axis cs:2,2) circle (1pt) node[below, xshift = -0.1cm]{$A(a\mid f(a)$};
					
					\draw [fill = black, color = black] (axis cs:4,2) node[below right]{$C$};
					
				\end{axis}
			\end{tikzpicture}
		\end{center}
	\end{frame}
	
	\begin{frame}[plain]
		\maketitle
	\end{frame}
	
	\begin{frame}
		\frametitle{Warum sollte man \LaTeX~benutzen?}
		\begin{itemize}
			\item hochwertige, professionelle Arbeiten
			\item schöner, sauberer Formelsatz
			\item einheitliche Formatierung
			\item Fokus liegt auf dem Inhalt
		\end{itemize}
	\end{frame}
	\begin{frame}{Befehle im Mathematikmodus}
		\begin{table}[h]
			\centering
			\begin{tabular} {l c c} \hline \hline 
				Ausdruck & Befehl & Ergebnis \\ \hline
				Bruch & \texttt{$\backslash$frac\{a\}\{b\}} & $\frac ab$ \\
				Wurzel & \texttt{$\backslash$sqrt[b]\{a\}} & $\sqrt[b]a$\\ Operatoren &\texttt{$\backslash$sin x}& $\sin x$ \\ >>mal-Punkt<< & \texttt{a$\backslash$cdot b} & $a\cdot b$\\ \hline \hline
		\end{tabular}
		\end{table}
	\end{frame}
	\begin{frame}{Das Haus vom Nikolaus}
		\begin{center}
			\begin{tikzpicture}[scale = 2]
			\draw (0,0) -- (0,1) -- (1,1) -- (1,0) -- (0,0) -- (1,1);
			\draw[blue, line width = 1pt] (1,0) -- (0,1);
			\draw[red, dashed] (0,1) -- (0.5, 2) node [above]{Spitze} -- (1,1);
		\end{tikzpicture}
		\end{center}	
	\end{frame}
\begin{frame}{Das Haus vom Nikolaus mit Kuppel}
	\begin{center}
		\begin{tikzpicture}[scale = 2]
			\draw[fill = black!30!white] (0,0) -- (0,1) -- (1,1) -- (1,0) -- (0,0) -- (1,1);
			\draw[blue, line width = 1pt] (1,0) -- (0,1);
			\draw[red, dashed] (0,1) -- (0.5, 2)  -- (1,1);
			\draw[orange, fill = orange] (0.5,2) circle (0.5);
		\end{tikzpicture}
	\end{center}	
\end{frame}
\begin{frame}{Winkel}
	\begin{center}
		\begin{tikzpicture}
			\draw[fill = red!30!white] (0,0) -- (1,0) arc[radius = 1, start angle = 0, end angle = 60] node[yshift = -10]{$\alpha$} -- (0,0) -- cycle;
			\draw (0,0) --(60: 3) (0,0) -- (3,0);
		\end{tikzpicture}
	\end{center}
\end{frame}
\end{document}